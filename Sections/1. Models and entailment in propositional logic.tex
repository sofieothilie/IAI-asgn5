\section{Models and entailment in propositional logic}
\subsection{Validity and Soundness}
\begin{large}

\begin{enumerate}[label=(\alph*)]
    \item Generate the vocabulary of the following argument.
    \item Translate the argument into propositional logic statements.
    \item Add a premise (P4) to make the conclusion of the argument valid.
\end{enumerate}
P1 to P3 are the premises, C is the conclusion:
\begin{itemize}
  \item (P1) If Peter’s argument is valid and all the premises of Peter’s argument are true, then Peter’s argument is sound.
  \item (P2) If the premises of Peter’s argument entail the conclusion of Peter’s argument, then Peter’s
argument is valid. 
  \item (P3) The premises of Peter’s argument entail the conclusion of Peter’s argument.
  \item (C) Peter’s argument is sound.
\end{itemize}


\end{large}


\subsection{Modelling}
\begin{large}
\raggedright A sentence is satisfiable if it is true in or satisfied by some model.

\begin{enumerate}[label=(\alph*)]
    \item $ (p \Rightarrow q) \Rightarrow (p \Rightarrow r) \Rightarrow ( q \Rightarrow r ) $ \\
    \begin{tabular}{c|c|c|c|c|c|c||c}
        p & q & r & $ p \Rightarrow q $ & $ p \Rightarrow r $ & $ q \Rightarrow r $ &
        $ (p \Rightarrow r) \Rightarrow ( q \Rightarrow r ) $ & Total \\
        \hline\hline
        0 & 0 & 0 & 1 & 1 & 1 & 1 & 1 \\
        \hline
        0 & 0 & 1 & 1 & 1 & 1 & 0 & 0\\
        \hline
        0 & 1 & 0 & 1 & 1 & 0 & 0 & 0 \\
        \hline
        0 & 1 & 1 & 1 & 1 & 1 & 1 & 1 \\
        \hline
        0 & 0 & 0 & 1 & 1 & 1 & 1 & 1 \\
        \hline
        0 & 0 & 1 & 1 & 1 & 1 & 1 & 1 \\
        \hline
        1 & 1 & 0 & 1 & 1 & 0 & 0 & 0 \\
        \hline
        1 & 1 & 1 & 1 & 1 & 1 & 1 & 1 \\
        \hline
        1 & 0 & 0 & 0 & 0 & 1 & 1 & 1 \\
        \hline
        1 & 0 & 1 & 0 & 1 & 1 & 1 & 1 \\
        \hline
        1 & 1 & 0 & 1 & 0 & 1 & 0 & 1 \\
        \hline
        1 & 1 & 1 & 1 & 1 & 1 & 1 & 1 \\
        \hline
    \end{tabular}

    \item $ (p \lor(\neg q \Rightarrow r)) \Rightarrow (q \lor (\neg p \Rightarrow r)) $ \\

    \begin{tabular}{c|c|c|c|c|c|c||c}
        p & q & r & $ \neg q \Rightarrow r $ & $ p \lor (\neg q \Rightarrow r) $ & $ \neg p \Rightarrow r $ &
        $ q \lor (\neg p \Rightarrow r) $ & Total \\
        \hline\hline
        0 & 0 & 0 & 1 & 1 & 1 & 1 & 1 \\
        \hline
        0 & 0 & 1 & 1 & 1 & 1 & 0 & 0\\
        \hline
        0 & 1 & 0 & 1 & 1 & 0 & 0 & 0 \\
        \hline
        0 & 1 & 1 & 1 & 1 & 1 & 1 & 1 \\
        \hline
        0 & 0 & 0 & 1 & 1 & 1 & 1 & 1 \\
        \hline
        0 & 0 & 1 & 1 & 1 & 1 & 1 & 1 \\
        \hline
        1 & 1 & 0 & 1 & 1 & 0 & 0 & 0 \\
        \hline
        1 & 1 & 1 & 1 & 1 & 1 & 1 & 1 \\
        \hline
        1 & 0 & 0 & 0 & 0 & 1 & 1 & 1 \\
        \hline
        1 & 0 & 1 & 0 & 1 & 1 & 1 & 1 \\
        \hline
        1 & 1 & 0 & 1 & 0 & 1 & 0 & 1 \\
        \hline
        1 & 1 & 1 & 1 & 1 & 1 & 1 & 1 \\
        \hline
    \end{tabular}
\end{enumerate} 
\end{large}

\subsection{Modelling 2}
\begin{large}
    
\end{large}