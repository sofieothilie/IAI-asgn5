\section{Resolution in propositional logic}
\subsection{Conjunctive Normal Form}
\begin{large}
    \textbf{Convert each of the following sentences to their Conjunctive Normal Form (CNF).
        A CNF formula is composed of a conjunction (AND) of one or more clauses, where each clause is a disjunction (OR) of literals.}
    \begin{enumerate}
        \item $ \mathbf{p \iff q} $

              $ (p \iff q) \equiv ((p \land q) \lor (\neg p \land \neg q)) \equiv \underline{\underline{((p \lor \neg q) \land (\neg p \lor q))}}$

        \item $ \mathbf{\neg ((p \Rightarrow q) \land r) }$

              $ (p \Rightarrow q ) \equiv  (p \land \neg q) $

              $ (( p \Rightarrow q) \land r) \equiv  (p \land \neg q \land r) $

              $ \neg ((p \Rightarrow q) \land r) \equiv $ \underline{\underline{$(\neg p \lor q \lor \neg r)$}}

        \item $\mathbf{ ((p \lor q) \lor (r \land \neg(q \Rightarrow r))) }$

              $\neg(q \Rightarrow r) \equiv \neg(q \land \neg r) \equiv (\neg q \lor r)$

              $ (r \land \neg(q \Rightarrow r)) \equiv (r \land (\neg q \lor r)) $

              $ ((p \lor q) \lor (r \land \neg(q \Rightarrow r))) \equiv ((p \lor q) \lor (r \land (\neg q \lor r))) $
              $\equiv \underline{\underline{((p \lor q \lor r) \land (p \lor r))}} $

        \item Yes, the expression above is in CNF\@.
    \end{enumerate}
\end{large}


\subsection{Inference in propositional logic}
\begin{large}
    \textbf{Use resolution to conclude r from the following statements.}
    \begin{enumerate}
        \item $ \mathbf{(p \Rightarrow q) \Rightarrow q }$
        \item $ \mathbf{p \Rightarrow r }$
        \item $ \mathbf{(r \Rightarrow s) \Rightarrow (\neg (s \Rightarrow q))} $
    \end{enumerate}
    Convert each statement into CNF\@:
    \begin{enumerate}
        \item $ ((p \Rightarrow q) \Rightarrow q) \equiv ((p \land \neg q) \lor q) $
        \item $ (p \Rightarrow r) \equiv (\neg p \lor r) $
        \item $ ((r \Rightarrow s) \Rightarrow (\neg (s \Rightarrow q))) \equiv ((s \land \neg q))$
    \end{enumerate}
    By performing resolution on clause 1 and 2 on q, we get:
    $ (p \land \neg q) \lor (\neg p \lor r)  $, which is equivalent to $ p \lor r $, which we will call clause 4 from here on.
    By performing resolution on clause 4 and 3 on q, we get:
    $ (s \land (\neg p \lor r)) $, which is equivalent to $ (s \land p) \lor (s \land r) $, which we will call clause 5 from here on.
    This result tells us that either $ p \lor r $ or s must be true.
    If $ p \lor r $ is true, we have $ (\neg p \lor r) $, which means r must be true.
    If s is true, then r may be either true or false.
\end{large}